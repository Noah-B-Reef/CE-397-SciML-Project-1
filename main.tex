\documentclass[12pt]{report}
\usepackage{scribe,graphicx,graphics}
\usepackage[english]{babel}
\usepackage{amsmath}
\usepackage{graphicx}
\usepackage{framed}
\usepackage[normalem]{ulem}
\usepackage{indentfirst}
\usepackage{amsmath,amsthm,amssymb,amsfonts}
\usepackage{cancel}
\usepackage{mathabx}
\usepackage[italicdiff]{physics}
\usepackage[T1]{fontenc}
%\usepackage{pifont} %For unusual symbols
%\usepackage{mathdots} %For unusual combinations of dots
\usepackage{wrapfig}
\usepackage{csquotes}
\usepackage{lmodern,mathrsfs}
\usepackage[inline,shortlabels]{enumitem}
\usepackage{float}
\usepackage{hyperref}
\usepackage[utf8]{inputenc}
\usepackage[english]{babel}
\usepackage{framed}
\usepackage[dvipsnames]{xcolor}
\usepackage{tcolorbox}
\setlist{topsep=2pt,itemsep=2pt,parsep=0pt,partopsep=0pt}
\usepackage{matlab-prettifier}
\usepackage[dvipsnames]{xcolor}
\usepackage[utf8]{inputenc}
\usepackage[a4paper, top=0.5in,bottom=0.2in, left=0.5in, right=0.5in, footskip=0.3in, includefoot]{geometry}
\usepackage[backend=biber,style=ieee]{biblatex}
\addbibresource{ref.bib}
\definecolor{codegreen}{rgb}{0,0.6,0}
\definecolor{codegray}{rgb}{0.5,0.5,0.5}
\definecolor{codepurple}{rgb}{0.58,0,0.82}
\definecolor{backcolour}{rgb}{0.95,0.95,0.92}

\lstdefinestyle{mystyle}{
    backgroundcolor=\color{backcolour},   
    commentstyle=\color{codegreen},
    keywordstyle=\color{magenta},
    numberstyle=\tiny\color{codegray},
    stringstyle=\color{codepurple},
    basicstyle=\ttfamily\footnotesize,
    breakatwhitespace=false,         
    breaklines=true,                 
    captionpos=b,                    
    keepspaces=true,                 
    numbers=left,                    
    numbersep=5pt,                  
    showspaces=false,                
    showstringspaces=false,
    showtabs=false,                  
    tabsize=2
}

\lstset{style=mystyle}
\newcommand{\norm[1]}{\left\lVert #1 \right\rVert}
\newcommand{\diam}{\text{diam}}
\newcommand{\nullspace}{\mathscr{N}}
\newcommand{\range}{\mathscr{R}}
\DeclareMathOperator{\rank}{\text{rank}}
\DeclareMathOperator{\Span}{\text{Span}}
\DeclareMathOperator{\nullity}{\text{nullity}}
\DeclareMathOperator*{\argmax}{arg\,max}
\DeclareMathOperator*{\argmin}{arg\,min}
\DeclareMathOperator*{\fl}{\text{fl}}
\course{CSE 397} 	
\coursetitle{SciML}	
\semester{Fall 2025}
\lecturer{} % Due Date: {\bf Mon, Oct 3 2016}}
\lecturetitle{Project Proposal}
\lecturenumber{}   
\lecturedate{}    
\input{commands.tex}

% Insert your name here!
\scribe{Student Name: Noah Reef}

\begin{document}


\maketitle

In \cite{Dahlbudding2024RayleighPINN} they look at using PINNS to perform radiative transfer modeling in exoplanetary atmospheres, specifically 
looking at a simplified 1D isotheremal model with pressure-dependent coefficients for absorption and Rayleigh scattering. They use two different PINNs, one to solve the absorption only case with given PDE,
\begin{equation}
    \frac{du}{dx} + \alpha'(x,y)u = 0 
\end{equation}
with $\alpha' = \ell_x \alpha_a$ and boundary conditions,
\begin{equation*}
    u(x=-1,y) = 1
\end{equation*}
and another to solve the full radiative transfer equation with Rayleigh scattering,
\begin{equation}
\cos(\phi) \frac{du}{dx} + \sin(\phi) \frac{du}{dy} + (\alpha_a + \alpha_s) \cdot u - \frac{\alpha_s}{4\pi} \cdot \int \Phi(\theta, \phi, \theta', \phi') u(\theta', \phi') d\Omega' = 0
\end{equation}
with boundary conditions,
\begin{equation*}
\begin{cases}
u\left(x =-1,y,|\phi| \leq \frac{\pi}{2}\right)&= \begin{cases}
1, |\phi| \leq \Delta_* \\ 0, \text{else} \end{cases}\\
u\left(x,y=1,|\phi| \leq 0\right)&= \begin{cases}
1, |\phi| \leq \Delta_* \\ 0, \text{else} \end{cases} \\
u\left(x = 1,y, |\phi| \geq \frac{\pi}{2}\right)&=0 \\
u_s(x,y=-1,\phi)&=0
\end{cases}
\end{cases}
\end{equation*}
They implement the BCs using soft-constraints for both PDEs\footnote{The Code is available at \url{https://github.com/DavidDahlbudding/AtmosphericScatteringPinn/tree/main/models}}. For this project I will implement hard constraints for the absorption only case, and compare the results to the soft-constraint implementation.
If time permits, I will also look at using a PINN to solve the full radiative transfer equation with hard constraints, and compare the results to the soft-constraint implementation.
\printbibliography[title={References}, heading=bibintoc]

\end{document}